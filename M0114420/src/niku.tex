
\section{研究の目的}
本研究では首元を被覆することにより、安心感を感
じさせるものを開発し、、実際に安心感を得ることがで
きたか検証を行う。


ネックピロー(図1)は、使用することで首の位置が安定し、
座った状態でも快適な居心地を得ることがてきる。しか
し、日常生活の中で、ネックピローを持ち歩くことは難
しい。そこで、衣服と一体となったピローの提案を行い、
より個人による快適なピローを実現するため、個人に
合わせた被覆の程度を調整できるものを実現する


\begin{figure b}
  \includegraphics[width=0.9\columnwidth]{pilow.png}
  \caption{\\図1:Studio Bananaが開発したネックピロー\\(OSTRICH PILLOW GO)}
  \label{fig:pilow}
\end{figure b}


%3
\section{関連研究}
\label{sec:format}

以下,情報処理学会論文誌ジャーナル用スタイルファイルを用いた論文フォーマッ
トの指針について述べるので,これに従って原稿を用意頂きたい.\LaTeX を用
いた一般的な文章作成技術については,\cite{okumura, companion} 等を参考に
されたい.

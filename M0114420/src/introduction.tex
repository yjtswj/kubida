\section{はじめに}
\label{sec:start}

首は動物における急所である。また、頭と胴体をつなぐ、人間の部位の中で唯一露出している部位の一つ
である。切られれば大量に出血し、喉頭隆起の部分を打たれると呼吸困難に陥る。


心理学において、喉元や首に触れるしぐさは、何らかの不安やストレス、心理的な動揺、驚異を感じていることの表れである。また、何らかのストレスを感じた時には、本能的に喉や首を守ろうとして手が伸びる。
喉や首を触っていると、急所をしっかり守っているということから安心感を得ることができる。


医学では、首の皮は薄く、首を冷やすと、冷たい血液が一気に体中に運ばれ、すぐに体が冷える。また首元を冷やすことで、肩こり、首周辺の筋肉が収縮。血流悪化により、首や肩に疲労物質が溜まり、首こりや肩こりを起こす。逆に、首は温めることで、


首に関する刑罰のひとつに、罪人の首元に刃物等により、頭部と胴体を切断する斬首刑がある。斬首刑では対象者は即死する。
切腹における介錯では、罪人の首を刀で斬り落とす。その際、首の骨の関節を切る、また「首の皮一枚」を残すなどいくつかの作法が存在する。
頭部を完全に切断せず首の皮で胴体に繋げた状態とするのは、胸の前にぶらさがった頭の重みで切腹者を前のめりの状態で死なせる配慮で、首が落ちずに危うくつながる意味「首の皮一枚」の由来である。

\section{事前実験}
\label{config}

どういった実装方法が適切か、ネックピローと衣服が一体となった場合、どのようなヴィジュアルになるのか検討を行うために事前実験を行った。ネックピローと衣服が一体となった場合の収納性を考慮し、空気式のネックピローを採用する。実際にネックピローの上に衣服を装着し、膨らませる前と、膨らませた後の状態を比較した。また様々な種類のネックピロー(図2)を装着することで安心感の要因を調査した。



\begin{figure b}
  \includegraphics[width=0.9\columnwidth]{ooip.jpg}
  \caption{ \\図2:様々な種類のネックピロー}
  \label{fig:ooip}
\end{figure b}


\section{実験結果}

ネックピローを膨らませる前の状態(図2)は、特に見た目に問題はなく、不快感はなかった。ネックピローを膨らませた後(図3)は、見た目が不恰好だが、居心地の良い安心感を感じた。安心感を感じさせる要因として、膨らんだネックピロー自体の圧力と、首を締め付ける圧力により首が安定し、居心地の良い安心感を感じさせることがわかった。

\begin{figure b}
  \includegraphics[width=0.9\columnwidth]{mae.png}
  \caption{ \\図3:実際にピローの上に衣服を装着し、ピローを膨らませる前の状態}
  \label{fig:mae}
\end{figure b}


\begin{figure b}
  \includegraphics[width=0.9\columnwidth]{ato.png}
  \caption{ \\図4:実際にピローの上に衣服を装着し、ピローを膨らませた後の状態}
  \label{ato:pilow}
\end{figure b}

\section{提案}

今回の実験を踏まえて、膨らんだネックピロー自体の圧力と、首を締め付ける圧力を調整できる、尚且つ衣服と一体になったネックピローを提案する。

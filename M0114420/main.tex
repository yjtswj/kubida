%%
%% 研究報告用スイッチ(情報処理学会用ファイルをEC2017用に変更)
%% [techreq]
%%
%% 欧文表記無しのスイッチ(etitle,jkeyword,eabstract,ekeywordは任意)
%% [noauthor]
%%

\documentclass[submit,techreq]{ec2017}
%\documentclass[submit,techreq,noauthor]{ipsj}

%% graphicパッケージ
\usepackage[dvipdfmx]{graphicx}
\graphicspath{{./figs/}}

\usepackage{latexsym}

\def\Underline{\setbox0\hbox\bgroup\let\\\endUnderline}
\def\endUnderline{\vphantom{y}\egroup\smash{\underline{\box0}}\\}
\def\|{\verb|}

\setcounter{巻数}{57}%vol53=2012
\setcounter{号数}{10}
\setcounter{page}{1}
\begin{document}
\title{首元の被覆による安心感の研究}
\etitle{}
\affiliate{IPSJ}{東京工科大学\\
1404-1 Katakuramachi Hachio-ji Tokyo 192-0914 JAPAN}


\paffiliate{JU}{東京工科大学\\
Johoshori Uniersity}

\author{南 慎也}{Minami Shinya}{IPSJ}[m011442061@edu.teu.ac.jp]


\begin{abstract}
首は動物における急所のひとつである。頭と胴体をつなぐ、人間の部位の中で唯一露出している部位の一つである。本稿では、首元を被覆することで安心感とネックピローについて着目し、衣服と一体となったネックピローを提案する。事前研究として、ネックピローの上に衣服の装着を行った。不快感はなく、安心感を与える要因としてピロー自体の圧力と、首を締め付ける圧力により、首が安定し、居心地の良い安心感を感じさせることがわかった。
\end{abstract}
\begin{jkeyword}
ウェアラブル、衣服、ネックピロー
\end{jkeyword}


\begin{eabstract}
This document is a guide to prepare a draft for submitting to IPSJ
Journal, and the final camera-ready manuscript of a paper to appear in
IPSJ Journal, using {\LaTeX} and special style files.  Since this
document itself is produced with the style files, it will help you to
refer its source file which is distributed with the style files.
\end{eabstract}

%\begin{ekeyword}
%IPSJ Journal, \LaTeX, style files, ``Dos and Dont's'' list
%\end{ekeyword}

\maketitle

%1
\section{はじめに}
\label{sec:start}

首は動物における急所である。首とは、頸部のことであり、人体において頭部と胴体をつなぐ部位である。人間の部位の中で唯一露出している部位の一つである。切られれば大量に出血し、喉頭隆起の部分を打たれると呼吸困難に陥る。

首は頭を自由な方向に動かせるため、大きな可動範囲を持っている。人は真後ろを剥くことができないが、フクロウ類は可動範囲が大きく、真後ろに首を向けることも、顔を上下逆さま近くの位置で曲げることもできる。

人の首が大きな可動範囲をもっていることは、それが異常であることがわかりやすいため、恐怖感や嫌悪感を感じさせる。例として、映画「エクソシスト」では悪魔憑き少女の首が真後ろを向き、日本では妖怪にろくろ首がある。(図1,2,3)

\begin{figure}[tbp]
  \begin{center}
    \includegraphics[width=0.9\columnwidth]{huku.jpg}
    \caption{顔を上下逆さまに曲げるフクロウ}
  \end{center}
\end{figure}

\begin{figure}[tbp]
  \begin{center}
    \includegraphics[width=0.9\columnwidth]{ekuso.jpg}
    \caption{映画「エクソシスト」悪魔憑きの少女}
  \end{center}
\end{figure}

\begin{figure}[tbp]
  \begin{center}
    \includegraphics[width=0.8\columnwidth]{rokuro.jpg}
    \caption{日本の妖怪のろくろ首}
  \end{center}
\end{figure}

心理学において、喉元や首に触れるしぐさは、何らかの不安やストレス、心理的な動揺、驚異を感じていることの表れである。また、何らかのストレスを感じた時には、本能的に喉や首を守ろうとして手が伸びる。
喉や首を触っていると、急所をしっかり守っているということから安心感を得ることができる。


医学では、首の皮は薄く、首を冷やすと、冷たい血液が一気に体中に運ばれ、すぐに体が冷える。また首元を冷やすことで、肩こり、首周辺の筋肉が収縮。血流悪化により、首や肩に疲労物質が溜まり、首こりや肩こりを起こす。逆に、首は温めることで、血行がよくなり、疲労回復、免疫力の向上や、首こりや肩こりの改善、美肌効果、頭痛改善などの効果が得られる。

首に関する刑罰のひとつに、罪人の首元に刃物等により、頭部と胴体を切断する斬首刑がある。斬首刑では対象者は即死する。2017年現在では、正式に刑罰である死刑の方法として、サウジアラビアのみが採用している。

切腹における介錯では、人は腹部を切開しただけでは即死しないため、切腹に際し、本人を即死させて負担と苦痛を軽減するするために、また即死できない本人が醜態を見せることのないよう、介助者が背後から罪人の首を刀で斬り落とす行為が行われた。その際、首の骨の関節を切断する。また「首の皮一枚」を残すなどいくつかの作法が存在している。
頭部を完全に切断せず首の皮で胴体に繋げた状態とするのは、胸の前にぶらさがった頭の重みで切腹者を前のめりの状態で死なせる配慮で、首が落ちずに危うくつながる意味「首の皮一枚」の由来である.


%2

\section{研究の目的}
本研究では首元を被覆することにより、安心感を感
じさせるものを開発し、、実際に安心感を得ることがで
きたか検証を行う。


ネックピロー(図1)は、使用することで首の位置が安定し、
座った状態でも快適な居心地を得ることがてきる。しか
し、日常生活の中で、ネックピローを持ち歩くことは難
しい。そこで、衣服と一体となったピローの提案を行い、
より個人による快適なピローを実現するため、個人に
合わせた被覆の程度を調整できるものを実現する


\begin{figure b}
  \includegraphics[width=0.9\columnwidth]{pilow.png}
  \caption{\\図1:Studio Bananaが開発したネックピロー\\(OSTRICH PILLOW GO)}
  \label{fig:pilow}
\end{figure b}


%3
\section{関連研究}
\label{sec:format}

以下,情報処理学会論文誌ジャーナル用スタイルファイルを用いた論文フォーマッ
トの指針について述べるので,これに従って原稿を用意頂きたい.\LaTeX を用
いた一般的な文章作成技術については,\cite{okumura, companion} 等を参考に
されたい.


%4
\section{事前実験}
\label{config}

どういった実装方法が適切か、ネックピローと衣服が一体となった場合、どのようなヴィジュアルになるのか検討を行うために事前実験を行った。ネックピローと衣服が一体となった場合の収納性を考慮し、空気式のネックピローを採用する。実際にネックピローの上に衣服を装着し、膨らませる前と、膨らませた後の状態を比較した。また様々な種類のネックピロー(図2)を装着することで安心感の要因を調査した。



\begin{figure b}
  \includegraphics[width=0.9\columnwidth]{ooip.jpg}
  \caption{ \\図2:様々な種類のネックピロー}
  \label{fig:ooip}
\end{figure b}


\section{実験結果}

ネックピローを膨らませる前の状態(図2)は、特に見た目に問題はなく、不快感はなかった。ネックピローを膨らませた後(図3)は、見た目が不恰好だが、居心地の良い安心感を感じた。安心感を感じさせる要因として、膨らんだネックピロー自体の圧力と、首を締め付ける圧力により首が安定し、居心地の良い安心感を感じさせることがわかった。

\begin{figure b}
  \includegraphics[width=0.9\columnwidth]{mae.png}
  \caption{ \\図3:実際にピローの上に衣服を装着し、ピローを膨らませる前の状態}
  \label{fig:mae}
\end{figure b}


\begin{figure b}
  \includegraphics[width=0.9\columnwidth]{ato.png}
  \caption{ \\図4:実際にピローの上に衣服を装着し、ピローを膨らませた後の状態}
  \label{ato:pilow}
\end{figure b}

\section{提案}

今回の実験を踏まえて、膨らんだネックピロー自体の圧力と、首を締め付ける圧力を調整できる、尚且つ衣服と一体になったネックピローを提案する。



%6
\section{今後の予定}

今後の予定としては、実装を行った後に、実際に安心感を感じることができるのか、また安心感や快適の程度を調査するための検証を行う。検証結果をもとに、再び実装を行う。首元の心地よさを追求し、安心感を創出するために、ネックピローを導電糸による発熱と、冷感接触効果を持つ毛糸と涼感クーリエで編まれた布を使用することによる冷却を実現し温度管理を行う。また温度センサーとギアドモーターによって、ネックピローの収縮の制御を行い、首元の温度管理の実装を行い、検証をする。空気を用いたネックピローの圧力の調整の他に、ネックピローの形を変える素材や、新しいクッション素材などによって、ネックピローへのアプローチを考案し、実装を行う。研究の応用として、体験型ゲームや体験型アトラクションにおいて、新しい体験を実現することができる活用方法を提案する。


%% 謝辞
\input{src/ack}

\begin{thebibliography}{10}

%\bibitem{latex}
%Lamport, L.: {\em A Document Preparation System \LaTeX User's Guide \&
%  Reference Manual}, Addison Wesley, Reading, Massachusetts (1986).
% (Cooke, E., et al.訳:文書処理システム \LaTeX,アスキー出版局
%  (1990)).

%\bibitem{total}
%伊藤和人: \LaTeX トータルガイド,秀和システムトレーディング (1991).
%\bibitem{nodera}
%野寺隆志:楽々 \LaTeX,共立出版 (1990).

\bibitem{}
Studio\,Banana,"OSTRICH\,PILLOW\,GO
\\"https://ostrichpillow.com
\bibitem{ac_wear}
weartronica,"Actuated\,Wear",Maker\,
\\Faire\,Tokyo\,2017,http://weartronica.com/
\bibitem{pairfeel}
佐藤玲美,海宝竜也,長谷川智洋,上田哲也,脇田玲
\\"Pair Feel:手をつなぐことで温度感覚を共有するマフラー", EC2014




\end{thebibliography}



%\pagebreak%%!!!
%\vspace*{-\baselineskip}%%!!!

%\appendix
%7
%\section{付録の書き方}

%付録がある場合には,参考文献リストの直後にコマンド \|\appendix| に引き続
%いて書く.付録では,\|\section| コマンドが{\bf A.1},{\bf A.2}などの見出
%しを生成する.

%7.1
%\subsection{見出しの例}

%付録の \|\subsetion| ではこのよう見出しになる.

%8
%\section{研究会論文用コマンド}
%\label{sig}

%各研究会論文誌(トランザクション)には各々に固有のサブタイトル,略称,通
%番がある.最終原稿では,以下のコマンドを \|\documentclass| の{\bf オプショ
%ン}とすることで,これらの情報を与える.

%\begin{itemize}
%\item \|PRO|(プログラミング)
%\item \|TOM|(数理モデル化と応用)
%\item \|TOD|(データベース)
%\item \|ACS|(コンピューティングシステム)
%\item \|CDS|(コンシューマ・デバイス\,\&\,システム)
%\item \|TBIO|(Bioinformatics)\footnote{%
%TBIO, SLDM, CVAは英文論文誌であるので和名はない.}
%\item \|SLDM|(System LSI Design Methodology)\footnotemark[5]
%\item \|CVA|(Computer Vision and Applicaitons)\footnotemark[5]
%\end{itemize}

%また英文論文作成の際には \|english| をオプションに追加すればよい.したがっ
%て,\|\documentclass[PRO]{ipsj}| とすれば「プログラミング」の和文用,
%\|\documentclass[PRO,english]| \|{ipsj}| とすれば英文用となる.

%また研究会には「号」と連動しない「発行月」があるため,学会あるいは編集委
%員会の指示に基づき,発行月を
%
%\begin{itemize}\item[]
%\|\setcounter{|{\bf 月数}\|}{<発行月>}|
%\end{itemize}
%
%によって指定する.

%この他,以下の各節で示すように,いくつかの論文誌に固有の機能を実現するた
%めのコマンドなどが用意されている.

%\newpage%%

%9
%\section{各分冊固有コマンド}

%各分冊によってそれぞれ細かい仕様が違うため,同じコマンドでも出力結果が異
%なる場合がある.また「再受付」,「再々受付」が入る場合があり,それらは

%\noindent
%和文では
%\begin{itemize}\item[]
%\|\|{\bf 再受付}\|{<年>}{<月>}{<日>}|\\
%\|\|{\bf 再再受付}\|{<年>}{<月>}{<日>}|
%\end{itemize}
%英文では
%\begin{itemize}\item[]
%\|\|{\bf rereceived}\|{<年>}{<月>}{<日>}|\\
%\|\|{\bf rerereceived}\|{<年>}{<月>}{<日>}|
%\end{itemize}
%とプリアンブルに追加する.

%9.1
%\subsection{\<「プログラミング(PRO)」固有機能}

%\<「論文誌:プログラミング」には論文以外に,プログラミング研究会での研究
%発表の内容梗概が含まれている.この内容梗概は,\|\documentclass|のオプショ
%ンとして\|abstract|を指定する.\ref{config}~節の\|\maketitle|までの内容
%からなるファイル(すなわち本文がないファイル)から生成する.なお\|\|{\bf
%受付}や\|\|{\bf 採録}は不要であるが,代わりに発表年月日を,

%\noindent
%和文では
%\begin{itemize}\item[]
%\|\|{\bf 発表}\|{<年>}{<月>}{<日>}|
%\end{itemize}
%英文では
%\begin{itemize}\item[]
%\|\|{\bf Presents}\|{<年>}{<月>}{<日>}|
%\end{itemize}
%により指定する.

%9.1
%\subsection{\<「データベース(TOD)」固有機能}

%\<「論文誌:データベース」の論文の担当編集委員は,
%\begin{itemize}\item[]
%\|\Editor{<氏名>}|
%\end{itemize}
%により指定する.和文では「担当編集委員」,英文では「Editor in Charge:」
%と入る.

%またスタイルの変更に伴い,\underline{本文の最後}に入るので,
%\|\end{document}|の前に直接置く.

%9.2
%\subsection{\<「コンシューマ・デバイス\,\&\,システム(CDS)」固有機能}

%\<「論文誌:コンシューマ・デバイス\,\&\,システム」では,
%論文の種類によって見出しが変わるため,
%オプションで切替えを行う.

%各種別は
%\begin{itemize}
%\item \|systems  |コンシューマ・システム論文\\
%\|         |Paper on Consumer Systems

%\item \|services |コンシューマ・サービス論文\\
%\|         |Paper on Consumer Services

%\item \|devices  |コンシューマ・デバイス論文\\
%\|         |Paper on Consumer Devices

%\item \|research |研究論文\\
%\|         |Research Paper
%\end{itemize}
%となる.

%和文のコンシューマ・システム論文なら,\\
%\|\documentclass[CDS,systems]{ipsj}|
%となり,英文原稿なら \|english|を追加すればよい.

%9.3
%\subsection{\<「Bioinformatics(TBIO)」固有機能}

%Trans.\ Bioinformatics (TBIO)は英文論文誌であるので,\|TBIO|オプションの
%指定によって自動的に\|english|オプションが指定されたものとみなされ,
%\|english| オプションの省略が可能.

%論文種別は以下の3種.
%\begin{itemize}
%\item \makebox[4.9zw][l]{指定なし} Original Paper (Default)
%\item \|Data     | Database/Software Paper
%\item \|Survey   | Survey Paper
%\end{itemize}

%\|\documentclass[TBIO]{ipsj}|でOriginal Paper,\\
%\|\documentclass[TBIO,Survey]{ipsj}|でSurvey Paperとなる.

%また,担当編集委員はTOD同様,\|\Editor|で定義するが,「Communicated by」
%となる.TOD同様,\|\end{document}|の前に直接置く.

%9.4
%\subsection{\<「Computer Vision and Applicaitons\\\<(CVA)」固有機能}

%Trans.\ CVAも英文論文誌であるため,\|english| オプションの省略が可.

%論文種別は3種類あり,
%\begin{itemize}
%\item \makebox[4.9zw][l]{指定なし} Regular Paper (Default)
%\item \|Research | Research Paper
%\item \|system   | Systems Paper
%\end{itemize}
%となる.

%TBIO同様,担当編集委員が入り,
%挿入文章もTBIO同様,「Communicated by」となる.

%9.5
%\subsection{\<「System LSI Design Methodology(SLDM)」固有機能}

%Trans.\ SLDMも英文論文誌であるため,\|english| オプションの省略が可.

%論文種別は2種類あり,
%\begin{itemize}
%\item \makebox[4.9zw][l]{指定なし} Regular Paper (Default)
%\item \|Short    | Short Paper
%\end{itemize}
%となる.

%SLDMも担当編集委員が入るが挿入文章が論文によって自動挿入文章が異なる.

%通常は「Recommended by Associate Editor:」,\|invited|のオプションが入っ
%た場合のみ,「Invited by Editor-in-Chief:」となる.



%% 以下は無視されます

\begin{biography}
\profile{m}{情報 太郎}{1970年生.1992年情報処理大学理学部情報科学科卒.
1994年同大大学院修士課程了.同年情報処理学会入社.オンライン出版の研究
に従事.電子情報通信学会,IEEE,ACM 各会員}
%
\profile{n}{処理 花子}{1960年生.1982年情報処理大学理学部情報科学科卒.
1984年同大大学院修士課程了.1987年同博士課程了.理学博士.1987年情報処
理大学助手.1992年架空大学助教授.1997年同大教授.オンライン出版の研究
に従事.2010年情報処理記念賞受賞.電子情報通信学会,IEEE,IEEE-CS,ACM
各会員}
%
\profile{s}{学会 次郎}{1950年生.1974年架空大学大学院修士課程了.
1987年同博士課程了.工学博士.1977年架空大学助手.1992年情報処理大学助
教授.1987年同大教授.2000年から情報処理学会顧問.オンライン出版の研究
に従事.2010年情報処理記念賞受賞.情報処理学会理事.電子情報通信学会,
IEEE,IEEE-CS,ACM 各会員}
%
\end{biography}



\end{document}

\section{はじめに}
\label{sec:start}

首は動物における急所である。首とは、頸部のことであり、人体において頭部と胴体をつなぐ部位である。人間の部位の中で唯一露出している部位の一つである。切られれば大量に出血し、喉頭隆起の部分を打たれると呼吸困難に陥る。

首は頭を自由な方向に動かせるため、大きな可動範囲を持っている。人は真後ろを剥くことができないが、フクロウ類は可動範囲が大きく、真後ろに首を向けることも、顔を上下逆さま近くの位置で曲げることもできる。

人の首が大きな可動範囲をもっていることは、それが異常であることがわかりやすいため、恐怖感や嫌悪感を感じさせる。例として、映画「エクソシスト」では悪魔憑き少女の首が真後ろを向き、日本では妖怪にろくろ首がある。(図1,2,3)

\begin{figure}[tbp]
  \begin{center}
    \includegraphics[width=0.9\columnwidth]{huku.jpg}
    \caption{顔を上下逆さまに曲げるフクロウ}
  \end{center}
\end{figure}

\begin{figure}[tbp]
  \begin{center}
    \includegraphics[width=0.9\columnwidth]{ekuso.jpg}
    \caption{映画「エクソシスト」悪魔憑きの少女}
  \end{center}
\end{figure}

\begin{figure}[tbp]
  \begin{center}
    \includegraphics[width=0.8\columnwidth]{rokuro.jpg}
    \caption{日本の妖怪のろくろ首}
  \end{center}
\end{figure}

心理学において、喉元や首に触れるしぐさは、何らかの不安やストレス、心理的な動揺、驚異を感じていることの表れである。また、何らかのストレスを感じた時には、本能的に喉や首を守ろうとして手が伸びる。
喉や首を触っていると、急所をしっかり守っているということから安心感を得ることができる。


医学では、首の皮は薄く、首を冷やすと、冷たい血液が一気に体中に運ばれ、すぐに体が冷える。また首元を冷やすことで、肩こり、首周辺の筋肉が収縮。血流悪化により、首や肩に疲労物質が溜まり、首こりや肩こりを起こす。逆に、首は温めることで、血行がよくなり、疲労回復、免疫力の向上や、首こりや肩こりの改善、美肌効果、頭痛改善などの効果が得られる。

首に関する刑罰のひとつに、罪人の首元に刃物等により、頭部と胴体を切断する斬首刑がある。斬首刑では対象者は即死する。2017年現在では、正式に刑罰である死刑の方法として、サウジアラビアのみが採用している。

切腹における介錯では、人は腹部を切開しただけでは即死しないため、切腹に際し、本人を即死させて負担と苦痛を軽減するするために、また即死できない本人が醜態を見せることのないよう、介助者が背後から罪人の首を刀で斬り落とす行為が行われた。その際、首の骨の関節を切断する。また「首の皮一枚」を残すなどいくつかの作法が存在している。
頭部を完全に切断せず首の皮で胴体に繋げた状態とするのは、胸の前にぶらさがった頭の重みで切腹者を前のめりの状態で死なせる配慮で、首が落ちずに危うくつながる意味「首の皮一枚」の由来である.

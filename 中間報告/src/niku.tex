
\section{研究の目的}
本研究では首元を被覆することにより、安心感を感
じさせるものを開発し、実際に安心感を得ることがで
きたか検証を行う。
 ネックピローは、使用することで首の位置が安定し、
座った状態でも快適な居心地を得ることがてきる。(図4)しか
し、日常生活の中で、ネックピローを持ち歩くことは難
しい。そこで、衣服と一体となったピローの提案を行い、
より個人による快適なピローを実現するため、個人に
合わせた被覆の程度を調整できるものを実現する。


\begin{figure}[tbp]
  \begin{center}
    \includegraphics[width=0.9\columnwidth]{pilow.png}
    \caption{Studio Bananaが開発したネックピロー\cite{OSTR} \\(OSTRICH PILLOW GO)}
    \label{fig:pilow}
  \end{center}
\end{figure}


%3
\section{関連研究}
\label{sec:format}

首元に関する研究として、他にどのような試みがあるのかを知る為に関連研究の調査を行った。
Actuated Wear\cite{ac_wear}はセンサー、ギアドモーター、および制御装置首周りの温度に応じて形が変化するネックウォーマーである。(図5)衣服下の温度として快適な32℃を保つよう、33℃を越えると緩み、31℃を下回ったら締まるようになっている。これにより屋内・屋外など状況の変化に応じて心地よさを保つことができる。(図6)

\begin{figure}[tbp]
  \begin{center}
    \includegraphics[width=0.9\columnwidth]{weartronica.jpg}
    \caption{Actuated Wear}
    \label{fig:pilow}
  \end{center}
\end{figure}

\begin{figure}[tbp]
  \begin{center}
    \includegraphics[width=0.9\columnwidth]{aw_motar.png}
    \caption{Actuated Wearのギアドモーター}
    \label{fig:pilow}
  \end{center}
\end{figure}

Pair Feel\cite{pairfeel}は手の接触を検知し、接触の有無によって双方が巻いたマフラーの温度が変化するウェアラブルデバイスである。(図7)親子やカップルの親密な関係において、手を繋ぐ行為はコミュニケーションや愛情表現の手段として行われることが多く、さらに、繋いだ者どうしの親密さがより高まることも期待される。発熱方法として、布に縫い付けた導電糸に通電を行う手法を採用している。また、発熱部は熱を感じやすい、うなじに触れる部分に装着している。(図8)これにより、カップルの親密さを向上させることができる。

\begin{figure}[tbp]
  \begin{center}
    \includegraphics[width=0.9\columnwidth]{pairfeel1.png}
    \caption{Pair Feel 全体構成図}
    \label{fig:pilow}
  \end{center}
\end{figure}

\begin{figure}[tbp]
  \begin{center}
    \includegraphics[width=0.9\columnwidth]{pairfeel2.png}
    \caption{Pair Feelの人体との接触部}
    \label{fig:pilow}
  \end{center}
\end{figure}

Actuated Wearでは首回りの温度に応じて形が変化するネックウォーマーによる心地よさを提案している。PairFeelでは手を繋ぎ、マフラーが温まることで、親子やカップルの親密さの向上を提案している。しかしどちらも着脱を行う必要があり、着用する場所や状況が限られている。以上の関連研究では首元の被覆を行い、その首元の温度の変化によって、居心地の良さや親密さの向上を提案をしている。首元を被覆するという行為の際に用いられるものとして、マフラーやネックウォーマーの他にも、ネックピローがあげられる。ネックピローは旅行や長時間の移動の際に、使用者の首の位置を安定させ、座った状態で快適な居心地を使用者に与えることができる。ネックピローは大きく分けて、クッションでできたタイプのものと、空気を入れて膨らませるエアークッションタイプのものがある。首を安定させる際のネックピローの中の空気圧や、クッションの素材による柔らかさや弾力感は、使用者の快適さや居心地の良さと密接な関係があるといる。本研究では、ネックピロー自体の圧力について着目し、ネックピローの圧力の変化を使用者によって調整を行い、また着脱を行わない衣服と一体となったネックピローを提案する。これにより、着脱を行うことなく、また着用する場所や状況を限定することなく、ネックピローによる快適で居心地の良い安心感を感じることができる。

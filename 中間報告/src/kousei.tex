


\section{事前実験}
\label{config}

どういった実装方法が適切か、ネックピローと衣服が一体となった場合、どのようなヴィジュアルになるのか検討を行うために事前実験を行った。ネックピローと衣服が一体となった場合の収納性を考慮し、空気式のネックピローを採用する。実際にネックピローの上に衣服を装着し、膨らませる前と、膨らませた後の状態を比較した。また様々な種類のネックピローを装着することで安心感の要因を調査した。(図9)

\begin{figure}[tbp]
  \begin{center}
    \includegraphics[width=1.0\columnwidth]{ooip.jpg}
    \caption{様々な種類のネックピロー}
    \label{fig:ooip}
  \end{center}
\end{figure}

\section{実験結果}

ネックピローを膨らませる前の状態は、特に見た目に問題はなく、不快感はなかった。(図10)ネックピローを膨らませた後は、見た目が不恰好だが、居心地の良い安心感を感じた。(図11)安心感を感じさせる要因として、膨らんだネックピロー自体の圧力と、首を締め付ける圧力により首が安定し、居心地の良い安心感を感じさせることがわかった。

\begin{figure}[tbp]
  \begin{center}
    \includegraphics[width=1.0\columnwidth]{mae.png}
    \caption{実際にピローの上に衣服を装着し、ピローを膨らませる前の状態}
    \label{fig:mae}
  \end{center}
\end{figure}

\begin{figure}[tbp]
  \begin{center}
    \includegraphics[width=1.0\columnwidth]{ato.png}
    \caption{実際にピローの上に衣服を装着し、ピローを膨らませた後の状態}
    \label{ato:pilow}
  \end{center}
\end{figure}

\section{提案}

今回の実験を踏まえて、膨らんだネックピロー自体の圧力と、首を締め付ける圧力を調整できる、なおかつ衣服と一体になったネックピローの提案を行う。これにより、着脱することなく、自分に合った圧力のピローで、どこでも快適で居心地の良い安心感を感じることができる。
